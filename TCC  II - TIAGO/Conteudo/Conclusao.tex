\chapter{Conclusões}

Com o avanço dos estudos na área de Inteligência
Artificial, os chatbots estão cada vez mais presentes e popularizados, sendo em
forma de serviço de Atendimento ao consumidor, em forma de comunicação e marketing ou até em formas mais
avançadas.


O presente trabalho visou a integração de um chatbot em pelo menos dois canais como estudo de caso para conversar e responder as perguntas de clientes interessados no mercado imobiliário, ou seja, compra e venda de imóveis. Uma das vantagens na utilização de chatbots em mais de um canal de comunicação é alta disponibilidade do serviço nos canais de preferência dos usuários finais. Além do mais, o curto tempo de resposta e a capacidade de interagir com milhões de usuários simultaneamente fazem dos chatbots uma ferramenta de grande utilidade comercial. 

Após o desenvolvimento do chatbot pode-se concluir, a partir da primeira tarefa de validação, que ele conseguiu responder e interagir em um nível em todos os canais propostos, que foram uma pagina web e o mensageiro slack, além do whatsapp que foi um canal adicional. Com a integração realizada nos canais de comunicação um chatbot pode, portanto, interagir com usuários em qualquer horário e com capacidade de manter um diálogo simultâneo com diversas pessoas.

Paralelamente, a tarefa de validação dois, que consistiu em testes de estresse, submeteu o chatbot a situações em que ocorreria milhares de requisições simultâneas para testar situações em que os limites do software estariam em extremo e avaliar seu comportamento. Neste ponto a integração do chatbot respondeu e se comportou adequadamente como esperado.

A integração foi implementada por meio das frameworks de desenvolvimento de código aberto Rasa e Bokit. Sendo assim, esse estudo verificou que é possível optar por uma alternativa de código aberto e grátis obtendo eficiência na integração em escala em oposição as soluções proprietárias e pagas. Entretanto, não existe bala de prata no desenvolvimento de software e o uso das frameworks adotadas neste trabalho não deve ser considerado uma regra.

Também ficou comprovado que uma vez que a base de conhecimento ou cérebro do chatbot está em funcionamento de acordo com a arquitetura é possível, com algumas modificações no conector, realizar a integração em outros canais. Destacando-se que a proposta de integração deste trabalho é flexível e não necessariamente precisa ser adotada para o desenvolvimento de chatbots. Entretanto, a arquitetura modularizada e coesa, um artefato gerado neste trabalho, deve ser adotada para evitar um alto acoplamento que pode causar problemas para evoluir o software. 



Para trabalhos futuros é necessário o desenvolvimento de adaptadores de código aberto que consiga realizar a comunicação do chatbot em mais canais de comunicação. Além disso, criar uma interface visual para que usuários não técnicos possam distribuir o chatbot nos canais disponíveis. 

