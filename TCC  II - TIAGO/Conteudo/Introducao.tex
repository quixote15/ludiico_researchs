\chapter{Introdução}

 O termo \textit{chatbot} vem do inglês onde \textit{chat} significa conversador e \textit{bot} é uma abreviação para \textit{robot} que significa robô. \citeonline{sganderla2003bonobot} definem \textit{chatbots}
como sistemas computacionais que simulam o comportamento humano em conversas e que são capazes de analisar, interpretar e responder perguntas.

Além disso, \citeonline{xu2017new} mostra que \textit{chatbots} são  um meio para o engajamento direto do usuário por meio de mensagens de texto para fins de atendimento ao cliente ou \textit{marketing}, evitando a necessidade de aplicativos ou páginas da \textit{Web} para propósitos especiais.

Percebendo essa capacidade de engajamento, \citeonline{brandtzaeg2017people} apontam que os \textit{chatbots} representam uma mudança potencial na forma como as pessoas interagem com os dados e serviços online. Esse potencial fez com que grandes empresas de tecnologia como Google \footnote{http://google.com}, Facebook \footnote{http://facebook.com} e Microsoft \footnote{http://microsoft.com} Vislumbrassem \textit{chatbots} como a próxima tecnologia que se tornará mais popular. Em 2016, a Google Assistant \footnote{https://assistant.google.com/intl/pt_br/} foi lançado como uma  assistente pessoal virtual capaz de realizar tarefas do dia-a-dia, como ligar para pessoas, mandar mensagens, pesquisar no Google, e ainda conversar com o usuário. Um ano depois, em 2017, aproximadamente 30 mil \textit{chatbots} foram lançados no Facebook Messenger \cite{brandtzaeg2017people}. A alexa \footnote{https://developer.amazon.com/pt-br/alexa} é o serviço de voz baseado em nuvem que está presente nos milhões de dispositivos Amazon\footnote{http://amazon.com} e que permite que desenvolvedores criem chatbots que forneçam aos usuários experiências de voz de uma maneira mais intuitiva. Por fim, a Microsoft também possui a Cortana\footnote{https://www.microsoft.com/pt-br/windows/cortana} que é uma assistente virtual inteligente do sistema operacional Windows 10, lançada em 2014. 

Em decorrência desse ganho de popularidade e importância, houve grandes avanços e aumento no número de \textit{frameworks}, também chamadas de plataformas, que facilitam o desenvolvimento desse tipo de aplicação. Uma framework ou plataforma para criação de chatbots fornece um conjunto de ferramentas que auxiliam e tornam o desenvolvimento e hospedagem de chatbots mais acessível e diminuem o tempo de implementação. Esse fatores tornam as frameworks a alternativa ideal para projetos que possuem restrições de tempo, pessoas, conhecimento e orçamento. O desenvolvimento de todos esses recursos como processamento de linguagem natural, controle de fluxos e integração com as interfaces humano-computador são partes complexas por si só e, portanto, desenvolver esses componentes do absoluto zero é custoso, improdutivo e pode fazer o projeto perder competitividade em um mundo globalizado e em constante mudança.

De acordo com o canal \citeonline{botsbrasil2019scratch}, o uso dessas \textit{frameworks} possibilita um nível considerável de abstração desses módulos complexos, permitindo assim um maior foco no desenvolvimento das interações e da base de conhecimento. Uma vez que o foco torna-se a construção do chatbot para melhor entender e atender os usuários, a integração do chatbot nos canais de comunicação, ou seja interfaces humano-computador, pode se tornar um gargalo dependendo da framework de desenvolvimento escolhida.

O número considerável de \textit{frameworks}, entretanto, torna difícil a tarefa de decidir qual será mais adequada integração em aplicações, bem como quais suas vantagens e desvantagens. Algumas \textit{frameworks} são destinadas exclusivamente a usuários sem conhecimento técnico e permitem a criação de \textit{chatbots} por meio de interfaces visuais. Todavia, tais soluções visuais possuem limitações e, portanto, só são viáveis para a criação de um \textit{chatbot} simples e sem muitos recursos. As plataformas Smooch\footnote{https://smooch.io}, FlowXO\footnote{https://flowxo.com},   MotionAI\footnote{http://www.motion.ai} são exemplos de plataformas pagas visuais que permitem a usuários comuns criar chatbots e integrar em múltiplos canais como páginas web e aplicativos móveis.

Algumas plataformas só permitem integração em um único canal, outras possuem integração em diversos canais porém cobrando pelo uso dos canais. As plataformas Chatfuel\footnote{https://chatfuel.com/}, ManyChat\footnote{https://manychat.com/} e Botsify\footnote{https://botsify.com/} só possuem comunicação com o Facebook Messenger. Para algumas aplicações esse tipo de abordagem atende as necessidades de projetos que só utilizarão esse único canal, entretanto, essas plataformas são proprietárias e pagas. O principal problema em utilizar plataformas proprietárias surge quando pretende-se escalar os recursos e habilidades dos \textit{chatbots}. O código fonte dessas plataformas não podem ser programáveis para atender aplicações específicas (não é possível modificar o módulo de processamento de linguagem natural com algoritmos customizados, por exemplo), o uso dos recursos é pago e os dados das aplicações podem ser monitorados por tais empresas. Além disso, algumas plataformas também são proprietárias da infraestrutura onde os \textit{chatbots} estão hospedados, gerando assim uma dependência ainda maior em termos de custo e tecnologia.

Por outro lado, existem \textit{frameworks} de código aberto que surgem como alternativa. \textit{Frameworks} como Rasa\footnote{https://rasa.com}, Botkit\footnote{https:botkit.ai} e Botpress\footnote{https://botpress.io} possibilitam maior flexibilidade já que são programáveis e permitem que o desenvolvedor hospede os \textit{chatbots} em qualquer infraestrutura escolhida. 

No entanto, decidir qual \textit{framework} será mais adequada  para um projeto de software não é tarefa trivial. Além do mais, como aponta \citeonline{brooks1987no} não existe bala de prata no desenvolvimento de software e, analogamente, não existe \textit{framework} ou plataforma que resolva todos os problemas. Em seu famoso artigo \textit{“No Silver Bullet — Essence and Accident in Software Engineering"} (Não há bala de prata — Essência e Acidente na Engenharia de Software em português), Frederick P. Brooks afirmou, em 1986, que não existe nenhuma metodologia de desenvolvimento ou técnicas de gestão que consiga aumentar significativamente a produtividade e confiabilidade da engenharia de software. Assim, de forma análoga, não existe a \textit{framework} perfeita e, portanto, a escolha da mesma deve levar em conta os requisitos e restrições do projeto em questão. 


% ----------------------------------------------------------
    % Nesse sentido, uma revisão sistemática foi realizada com o intuito de elencar as \textit{frameworks} mais utilizadas e avaliar suas características e funcionalidades. Esse levantamento pode servir como referência para trabalhos futuros na área.  

% ----------------------------------------------------------


Portanto, visto que o desenvolvimento de chatbots é realizado com o uso de alguma framework, a escolha da \textit{framework} deve levar em conta os canais de comunicação disponíveis para integração e possibilidade de integração em mais de um canal simultanêamente já que, segundo \citeonline{strutzel2015presencca}, a presença digital empresas e instituições são essenciais para desenvolver estratégias de marketing, vendas e relacionamento com clientes. A assistente virtual do Bradesco, a Bia\footnote{https://banco.bradesco/html/classic/promocoes/bia/para-voce.shtm}, é um exemplo que interface conversacional que está disponível aos clientes bradesco no whatsapp, aplicativo do banco, assistente do google e sms.  Nesse sentido, o uso de \textit{chatbots} em múltiplos canais de comunicação torna-se uma ferramenta de comunicação, otimização de tarefas e recursos que fornece maior visibilidade, flexibilidade e escalabilidade. 


A maioria das abordagens de desenvolvimento utiliza, por limitação técnica ou da plataforma, um tipo específico de canal de comunicação. Como apontado anteriormente, existiam 30 mil \textit{chatbots} que executam no Facebook Messenger em 2017, não podendo, porém, ser reutilizados em outros canais de comunicação como páginas web, aplicativos ou outros mensageiros como Telegram, Slack e Whatsapp. Outra solução foi desenvolvida por \citeonline{jia2004study} cujo trabalho foi a construção de um \textit{chatbot} com processamento de linguagem natural para o ensino de línguas na web. No Brasil, o site JusBrasil \footnote{www.jusbrasil.com.br} utiliza um chatbot que ajuda o cidadão a encontrar advogados e resolver problemas jurídicos. 


O presente trabalho tem como
objetivo analisar que aspectos devem ser
considerados para realizar integração de chatbots em mais de um canal de comunicação por meio de frameworks de desenvolvimento. Sendo assim, esse estudo se justifica por buscar identificar como
o uso das frameworks pode gerar eficiência na integração em escala para empresas e para clientes por meio de um estudo de caso que consiste na integração de um chatbot no domínio de imóveis que estará presente em pelo menos dois canais, uma pagina web e o Slack. A extensibilidade da integração proposta para outros canais e seu grau de dificuldade também é um problema abordado neste trabalho.
 





% PARA REFLEXÃO

\section{Objetivos}

O principal objetivo a proposta e análise de integração da tecnologia chatbot para sistemas em desenvolvimento em pelo menos dois canais de comunicação.

\subsection{Objetivos específicos}

\begin{itemize}

    \item Criar e analisar abordagem de integração da tecnologia chatbot na interface web;
    \item  Criar e analisar abordagem de integração da tecnologia chatbot no mensageiro Slack;
    \item Mapear aspectos relevantes para tomada decisória sobre o uso de framewoks para integração de chatbot nesses tipos de interfaces;
    
    \item Mapear aspectos relevantes para tomada decisória sobre o uso de framewoks para desenvolvimento da tecnologia chatbot;
    
\end{itemize}

\subsection{Metodologia}

Este trabalho é classificado como Pesquisa Aplicada porque envolve a geração de conhecimento para aplicação prática e direcionado para um problema específico. Os passos deste trabalho são:


\begin{enumerate}
    \item \textbf{Determinação do tema - problema}: Nesta fase, verificou-se a necessidade de um pouco
exploração do contexto mercadológico para o início das atividades e determinação da real necessidade de resolução do problema.

\item \textbf{Refletir sobre o uso de framework vs implementação própria}: Nesta fase,  verificou-se que o uso de frameworks é a abordagem mais utilizada pois diminui o tempo desenvolvimento permitindo assim mais competitividade, reuso e maior foco na criação de diálogos e base de conhecimento dos chatbots. Também ficou constatado que os levantamentos subsequentes como revisão da literatura e mercadológica deveria ser voltado para essas frameworks pois as integrações de chatbots nas interfaces humano-computador acontecem hegemonicamente por meio das mesmas. 

    \item \textbf{Levantamento bibliográfico}: Nesta etapa foi realizada uma Revisão Sistemática da literatura seguindo o procedimento proposto por \cite{kitchenham2004procedures} para identificar e analisar trabalhos que proponham alguma \textit{framework} para auxiliar no desenvolvimento de \textit{chatbots}, porém ficou constatado a falta de tais artefatos na literatura e, portanto, uma busca no mercado se fez de suma importância.
    \item \textbf{Pesquisa de mercado}: Nesta etapa, verificou-se a necessidade de exploração das \textit{frameworks} utilizadas no mercado devido a falta de resultados na etapa de levantamento bibliográfico. Nesta etapa  também foi determinada a \textit{framework} para desenvolvimento de \textit{chatbots} mais adequada para este trabalho a partir de uma análise comparativa de características e restrições de projeto.
    
        
    \item \textbf{Identificação do estudo de caso}: Nesta fase, foi identificado e selecionado um chatbot que possui o domínio na área de imóveis para realizar a integração.
    
    \item \textbf{Análise dos Requisitos}: Foi utilizado a técnica de estória de usuário para levantar os requisitos deste trabalho de maneira ágil e intuitiva. A partir disso, atribuiu-se pontos de acordo com a dificuldade das estórias.

    \item \textbf{Desenvolvimento -  Arquitetura do sistema de integração}: Nesta etapa foi realizada a criação e integração de um \textit{chatbot} em uma pagina web e Slack. A arquitetura do sistema foi modelada utilizando o protocolo Http e webhooks para que funcione sem depender de tecnologias específicas. Além disso, foi feita a análise das entidades, intenções e contexto em que um \textit{chatbot} se aplica para que pudesse ser feita uma integração em nos canais propostos. 
    
    \item \textbf{Análise técnico-científica da integração}: Nessa fase, foram realizados testes funcionais e análise dos resultados.
\end{enumerate}

\section{Estrutura do Documento}

Para facilitar a navegação e melhor entendimento, este documento está
estruturado em capítulos e seções, que são:
\begin{itemize}
\item {Capítulo 1 - Introdução}
\item {Capítulo 2 - Fundamentação teórica} \cite{Cormen:2009};
\item {Capítulo 3 - Revisão sistemática}: \cite{Weicker:1984:DSS:358274.358283}
\item {Capítulo 5 - Resultados}: \cite{Linux:402081};
\item {Capítulo 6 - Conclusão}: \cite{SBC:2012};
\end{itemize}
