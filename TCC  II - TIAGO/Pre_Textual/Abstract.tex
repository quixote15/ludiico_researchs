% resumo em inglês
\setlength{\absparsep}{18pt} % ajusta o espaçamento dos parágrafos do resumo
\begin{resumo}[Abstract]
 \begin{otherlanguage*}{english}
   
The growing popularization of the use of conversational interfaces, also called chatbots, by companies, technical and non-technical users has paved the way for the emergence of several frameworks and platforms focused on the development of this type of software. Due to the large number of these frameworks, deciding which software development to use is no trivial task since there is no silver bullet in software engineering. As a result, a survey of the main frameworks and their characteristics was performed in order to facilitate this decision making. The human-computer interfaces, also called communication channels, on which chatbots will be available, such as messengers and web pages, are another important consideration. In 2017, around 30,000 chatbots were available on Facebook and that number just grows. However, these and other chatbots are only available on a single communication channel, and companies in many industries often have a digital presence and more than one channel. Therefore, a case study was conducted where integrating a chatbot for real estate  in at least two communication channels, the Slack messenger and a web page. After integration into the two proposed channels, the integration extension for WhatsApp Messenger was performed.

 
   \textbf{Keywords}: Chatbots, Conversational Interface, Framework, Artificial Inteligence.
 \end{otherlanguage*}
\end{resumo}