% resumo em português
\setlength{\absparsep}{18pt} % ajusta o espaçamento dos parágrafos do resumo
\begin{resumo}
 A crescente popularização do uso de interface conversacionais, também chamadas de chatbots, por empresas, usuários técnicos e não técnicos abriu caminho para o surgimento de diversas frameworks e plataformas voltadas para o desenvolvimento desse tipo de software. Devido ao grande número dessas frameworks a decisão sobre a qual utilizar para o desenvolvimento de software não é tarefa trivial já que não existe bala de prata na engenharia de software. Em decorrência disso, um levantamento das principais frameworks e suas características foi realizado com o intuito de facilitar essa tomada de decisão. As interfaces humano-computador, também chamadas de canais de comunicação, em que os chatbots estarão disponíveis, como mensageiros e paginas web, são outro aspecto importante a se considerar. Em 2017, cerca de 30 mil chatbots estavam disponíveis no Facebook e esse número só cresce. Entretanto, esses e outros chatbots só estão disponíveis em um único canal de comunicação e frequentemente empresas de diversos setores possuem uma presença digital em mais de um canal. Portanto, um estudo de caso foi realizado onde a integração de um chatbot para o ramo imobiliário em pelo menos dois canais de comunicação, o mensageiro Slack e uma página web. Após a integração nos dois canais propostos foi realizada a extensão da integração para o mensageiro Whatsapp. 

 \textbf{Palavras-chave}: Chatbots, interface conversacional, frameworks, Inteligência artificial.
\end{resumo}