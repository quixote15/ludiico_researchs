
@article{ ISI:000442980500006,
Author = {Ma, Shang-Pin and Ho, Ching-Ting},
Title = {{Modularized and Flow-Based Approach to Chatbot Design and Deployment}},
Journal = {{JOURNAL OF INFORMATION SCIENCE AND ENGINEERING}},
Year = {{2018}},
Volume = {{34}},
Number = {{5, SI}},
Pages = {{1187-1201}},
Month = {{SEP}},
Abstract = {{Chatbots are computer programs designed to chat with users via text or
   voice through the use of techniques of Web services, data analysis, and
   artificial intelligence (AI). Currently, the use of chatbot is becoming
   an important trend in the field of data science. An increasing number of
   chatbots are being built on social platforms, such as Facebook, LINE,
   and Slack. This has led to the development of numerous tools and online
   platforms for the construction of chatbots; however, most of these
   services do not provide comprehensive support for the visual
   representation and control of conversational flows, bi-directional Web
   service integration, or the systematic reuse of conversations.
   Developing a complex chatbot with external Web services requires the
   writing of extensive conversation scripts and additional coding. In this
   paper, we propose a visual, flow-based approach to the construction of
   chatbots on the Node-RED platform, referred to as FCF (Flow-based
   Chatbot Framework). This system is based on the newly-devised data
   format for Webhook, thereby allowing bidirectional service integration
   for software applications. Five chatbot dialogue patterns and three
   chatbot application scenarios are provided to be components for the
   construction of complex chatbot applications.}},
DOI = {{10.6688/JISE.201809\_34(5).0005}},
ISSN = {{1016-2364}},
Unique-ID = {{ISI:000442980500006}},



possiveis

doi={10.1109/SIBCON.2015.7147079}, 
ISSN={}, 
month={May},}
@INPROCEEDINGS{5167660, 
author={S. J. du Preez and M. Lall and S. Sinha}, 
booktitle={IEEE EUROCON 2009}, 
title={An intelligent web-based voice chat bot}, 
year={2009}, 
volume={}, 
number={}, 
pages={386-391}, 
abstract={This paper presents the design and development of an intelligent voice recognition chat bot. The paper presents a technology demonstrator to verify a proposed framework required to support such a bot (a Web service). While a black box approach is used, by controlling the communication structure, to and from the Web-service, the Web-service allows all types of clients to communicate to the server from any platform. The service provided is accessible through a generated interface which allows for seamless XML processing; whereby the extensibility improves the lifespan of such a service. By introducing an artificial brain, the Web-based bot generates customized user responses, aligned to the desired character. Questions asked to the bot, which is not understood is further processed using a third-party expert system (an online intelligent research assistant), and the response is archived, improving the artificial brain capabilities for future generation of responses.}, 
keywords={expert systems;speech recognition;Web services;XML;intelligent Web-based voice chat bot;intelligent voice recognition chat bot;Web service;black box approach;XML processing;third-party expert system;online intelligent research assistant;XML;Speech recognition;Web services;Simple object access protocol;Expert systems;Artificial intelligence;Feeds;Signal processing;Web server;Paper technology;AI;XML;JAVA;AIML;ALICE}, 

doi={10.1109/PHEALTH.2009.5754825}, 
ISSN={}, 
month={June},}
@INPROCEEDINGS{5254226, 
author={J. M. Mendes and A. Bepperling and J. Pinto and P. Leitao and F. Restivo and A. W. Colombo}, 
booktitle={2009 33rd Annual IEEE International Computer Software and Applications Conference}, 
title={Software Methodologies for the Engineering of Service-Oriented Industrial Automation: The Continuum Project}, 
year={2009}, 
volume={1}, 
number={}, 
pages={452-459}, 
abstract={Service-orientation represents a new wave of features and solutions by bringing closer information technology to the industrial domain, particularly factory shop floors. The service-oriented automation software entities (designated here by bots) used in such approach requires a short set of methodologies and software targeting their specification for both computer systems and embedded automation devices. The present work explains the adopted methodologies and software developments for the engineering of service-based automation systems. The main contents focus on the specification of a framework for the development of bots and supporting engineering tools that are part of the Continuum project. The paper also does an overview over the engineering steps from the system design to the operation, and focuses the importance of the maintenance of automation bots. Such applications will contribute to decrease the development time and reduce the components' interdependency, offering enough flexibility for automatic reconfiguration of shop-floor layouts.}, 
keywords={formal specification;industrial control;industrial robots;mobile robots;object-oriented programming;production facilities;project management;software architecture;software maintenance;Web services;software engineering methodology;service-oriented industrial automation;continuum project development tool;information technology;factory shop floor layout;software specification;embedded automation device;computer system design;software development;distributed software component interdependency;automatic reconfiguration;service-oriented architecture;automation bot;autonomous robot;software maintenance;Computer industry;Manufacturing automation;Design automation;Embedded software;Information technology;Production facilities;Software design;Embedded computing;Programming;Design engineering;Software engineering;Service-oriented Architectures;Industrial Automation}, 

@INPROCEEDINGS{6806005, 
author={M. Hijjawi and Z. Bandar and K. Crockett and D. Mclean}, 
booktitle={2014 6th International Conference on Computer Science and Information Technology (CSIT)}, 
title={ArabChat: An Arabic Conversational Agent}, 
year={2014}, 
volume={}, 
number={}, 
pages={227-237}, 
abstract={This paper details the development of a novel and practical Conversational Agent for the Arabic language called ArabChat. A conversational Agent is a computer program that attempts to simulate conversations between machine and human. In this paper, the term `conversation' or `utterance' refers to real-time chat exchange between machine and human. The proposed framework for developing the Arabic Conversational Agent (ArabChat) is based on Pattern Matching approach to handle users' conversations. The Pattern Matching approach is based on the matching process between a user's utterance and pre-scripted patterns that represents different topics organized through novel scripting structure. A real experiment has been done in Applied Science University in Jordan as an information point advisor for their native Arabic students to evaluate the ArabChat.}, 
keywords={linguistics;pattern matching;software agents;ArabChat;Arabic conversational agent;Arabic language;real-time chat exchange;Arabic Conversational Agent;pattern matching;Applied Science University;Jordan;Context;Pattern matching;Engines;Natural language processing;Fires;Indexes;Firing;Conversational agent;chatterbot;Arabic and scripting}, 
doi={10.1109/IMIS.2012.114}, 
ISSN={}, 
month={July},}
@INPROCEEDINGS{6385727, 
author={R. Qiu and Z. Ji and A. Noyvirt and A. Soroka and R. Setchi and D. T. Pham and S. Xu and N. Shivarov and L. Pigini and G. Arbeiter and F. Weisshardt and B. Graf and M. Mast and L. Blasi and D. Facal and M. Rooker and R. Lopez and D. Li and B. Liu and G. Kronreif and P. Smrz}, 
booktitle={2012 IEEE/RSJ International Conference on Intelligent Robots and Systems}, 
title={Towards robust personal assistant robots: Experience gained in the SRS project}, 
year={2012}, 
volume={}, 
number={}, 
pages={1651-1657}, 
abstract={SRS is a European research project for building robust personal assistant robots using ROS (Robotic Operating System) and Care-O-bot (COB) 3 as the initial demonstration platform. In this paper, experience gained while building the SRS system is presented. A main contribution of the paper is the SRS autonomous control framework. The framework is divided into two parts. First, it has an automatic task planner, which initialises actions on the symbolic level. The planner produces proactive robotic behaviours based on updated semantic knowledge. Second, it has an action executive for coordination actions at the level of sensing and actuation. The executive produces reactive behaviours in well-defined domains. The two parts are integrated by fuzzy logic based symbolic grounding. As a whole, they represent the framework for autonomous control. Based on the framework, several new components and user interfaces are integrated on top of COB's existing capabilities to enable robust fetch and carry in unstructured environments. The implementation strategy and results are discussed at the end of the paper.}, 
keywords={fuzzy logic;fuzzy set theory;home automation;human-robot interaction;intelligent robots;mobile robots;service robots;telerobotics;user interfaces;robust personal assistant robot;SRS project;European research project;ROS;robotic operating system;Care-O-bot 3;COB 3;SRS autonomous control;automatic task planner;action initialisation;proactive robotic behaviour;semantic knowledge;coordination action;reactive behaviour;fuzzy logic-based symbolic grounding;autonomous control;user interfaces;unstructured environments;real-time sensing;real-time actuation;Robot kinematics;Robot sensing systems;Planning;Semantics;Grounding;Electronic mail}, 